% Apresentação do trabalho de graduação

\documentclass{beamer}
\usepackage[brazil]{babel}
%\usetheme{Warsaw} % tema do slide, o azul e preto no fundo
%\usetheme{Berlin}
%\usetheme{EastLansing}
%\usetheme{Luebeck}
\usetheme{Copenhagen}
%\usetheme{Szeged}
%\usetheme{Singapore} %gostei
%\usetheme{AnnArbor}%nenhuma  cor fica legal
%\usetheme{CambridgeUS}%bom modelo
%\usetheme{Boadilla}%Muito bom
%\usetheme{Madrid}%excelente tema
%\usetheme{Boxes}
%\usetheme{Warsaw} % tema do slide, o azul e preto no fundo
%\usecolortheme{seahorse}
%\usecolortheme{beaver}
%\usecolortheme{albatross}
%\usecolortheme{crane}
%\usecolortheme{dolphin}%bom esquema de cor
%\usecolortheme{dove}
%\usecolortheme{fly}
%\usecolortheme{lily}
%\usecolortheme{orchid}%ótimo com Cambridge
%\usecolortheme{rose}
\usecolortheme{seagull}
%\usecolortheme{sidebartab}
%\usecolortheme{structure}
%\usecolortheme{whale}
%\usecolortheme{wolverine}
\usepackage{hyperref}
\usepackage{graphicx}
\usepackage{ragged2e}
\usepackage{latexsym}
%\usepackage[latin1]{inputenc}
\usepackage[utf8]{inputenc} % esse serve pra digitar com acento normal
\usepackage{amsmath}
\usepackage{amsfonts}
\usepackage{amssymb}

\useoutertheme[subsection=false,shadow=default]{miniframes}
\useinnertheme{default}
\setbeamertemplate{footline}{}
%\beamertemplatenavegationsymbolsempty{}

%\setlength{\parindent}{-10cm} 

\title[Defesa]{Implementação de Suporte a Texturas 3D para Renderização de
Volumes no Blender}
\author[Rafael Campos]{Rafael Cerqueira de Campos \\ Orientador: Prof. Dr. Mario Augusto de Souza Liziér}
\institute[DC - UFSCAR]{Graduação em Engenharia de Computação - UFSCar}
\date{\today}
\thispagestyle{empty}
\subject{alguma coisa}

%%Define o caminho das figuras, válido somente para o comando \includegraphics
\graphicspath{{images/}}

\begin{document}


%Primeira página da apresentação
\frame{\titlepage}

\AtBeginSubsection[]

%%%%%%%%%%%%%%%%%%%%%%%%%%%%%%%%%%%%%%%%%%%%%%%%%%%%%%
%Sum\'ario
\begin{frame}[allowframebreaks]
\tableofcontents
\end{frame}
%%%%%%%%%%%%%%%%%%%%%%%%%%%%%%%%%%%%%%%%%%%%%%%%%%%%%%

%%%%%%% Primeiro Slide %%%%%%%%%%%%%%%%%%%%%%%%%%%%%%%%%%%%%%%%%
\section{Introdu\c{c}\~ao}

\begin{frame}

\frametitle{Introdu\c{c}\~ao:}

\begin{itemize}

%\item Objetivo: Resolver o Problema de Minimiza\c{c}\~ao com Restri\c{c}\~oes Lineares;
\item \'E frequente nas aplica\c{c}\~oes problemas em que as derivadas n\~ao est\~ao dispon\'iveis;
\item H\'a in\'umeras aplica\c{c}\~oes cujas restri\c{c}\~oes s\~ao apenas lineares;
\item Subproblema em m\'etodos voltados para problemas gerais;
%\item Em Otimiza\c{c}\~ao com derivadas temos v\'arios m\'etodos espec\'ificos para resolver problemas com Restri\c{c}\~oes apenas lineares;
%\item Problemas de Otimiza\c{c}\~ao em que as derivadas n\~ao podem ser calculadas tem aparecido com certa frequ\^encia;
\item Estudar e propor m\'etodos sem derivadas para resolv\^e-los;
\item Focamos nosso trabalho no M\'etodo de Busca Padr\~ao aplicado ao problema com restri\c{c}\~oes lineares;
\item Propomos novas estrat\'egias de busca e um novo Padr\~ao buscando melhorar o desempenho do m\'etodo.

\end{itemize}

\end{frame}
%%%%%%%%%%%%%%%%%%%%%%%%%%%%%%%%%%%%%%%%%%%%%%%%%%%%%%%%%	

\end{document}